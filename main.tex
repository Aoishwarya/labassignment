\documentclass{article}
\usepackage[utf8]{inputenc}
\usepackage{graphicx}
\usepackage{tabularx}
\usepackage{mathtools}
\usepackage{color}
\usepackage{}
\title{\textbf{Data Structure : Theoretical Approach}}

\author{Aoishwarya Mojumder\\Department of Information and Communication Engineering\\Noakhali Science and Technology University}


\begin{document}


\maketitle
\begin{center}
    \includegraphics[width=.60\textwidth]{20210917_122345.png}
\end{center}

\section{\textbf{\color{blue}INTRODUCTION:}}
Data structures serve as the basis for abstract data 
types (ADT). "The ADT defines the logical form of 
the data type. The data structure implements the 
physical form of the data type."Different types of data 
structures are suited to different kinds of applications, 
and some are highly specialized to specific tasks. For 
example, relational databases commonly use B-tree 
indexes for data retrieval, while compiler 
implementations usually use hash tables to look up 
identifiers. Data structures provide a means to manage 
large amounts of data efficiently for uses such as large 
databases and internet indexing services. Usually, 
efficient data structures are key to designing efficient 
algorithms. Some formal design methods and 
programming languages emphasize data structures, 
rather than algorithms, as the key organizing factor in 
software design. Data structures can be used to 
organize the storage and retrieval of information 
stored in both main memory and secondary memory. 
Data structures are generally based on the ability of a computer to fetch and store data at any place in its 
memory, specified by a pointer—a bit string, 
representing a memory address, that can be itself 
stored in memory and manipulated by the program. 

The implementation of a data structure usually 
requires writing a set of procedures that create and 
manipulate instances of that structure. The efficiency 
of a data structure cannot be analyzed separately from 
those operations. This observation motivates the 
theoretical concept of an abstract data type, a data 
structure that is defined indirectly by the operations 
that may be performed on it, and the mathematical 
properties of those operations (including their space 
and time cost).[citation needed]An array is a number 
of elements in a specific order, typically all of the 
same type (depending on the language, individual 
elements may either all be forced to be the same type, 
or may be of almost any type). Elements are accessed 
using an integer index to specify which element is 
required. Typical implementations allocate contiguous 
memory words for the elements of arrays (but this is 
not necessity). Arrays may be fixed-length or 
resizable. A linked list (also just called list) is a linear 
collection of data elements of any type, called nodes, 
where each node has itself a value, and points to the 
next node in the linked list. The principal advantage 
of a linked list over an array, is that values can always 
be efficiently inserted and removed without relocating 
the rest of the list. Certain other operations, such as 
random access to a certain element, are however 
slower on lists than on arrays.


\section{\textbf{\color{blue}ABSTRACT}}
Run with accordance with significance. The first if 
these this paper explains about the basic terminologies 
used in this paper in data structure. Better running 
times will be other constraints, such as memory use 
which will be paramount. The most appropriate data 
structures and algorithms rather than through hacking 
removing a few statements by some clever coding. 
Data structures serve as the basis for abstract data 
types (ADT). "The ADT defines the logical form of 
the data type. The data structure implements the 
physical form of the data type."Different types of data 
structures are suited to different kinds of applications, 
and some are highly specialized to specific tasks. For 
example, relational databases commonly use B-tree 
indexes for data retrieval, while compiler 
implementations usually use hash tables to look up 
identifiers.




\section{\textbf{\color{blue}Depth of node}}
The depth of node is the length of the path from the 
root to the node. A rooted tree with only one node has 
a depth of zero.
\begin{center}
\includegraphics[width=.60\textwidth]{Tree depth.jpg}
\end{center}


The height of a node in a binary tree is the largest number of edges in a path from a leaf node to a target node. The height of a binary tree is the height of the root node in the whole binary tree. In other words, the height of a binary tree is equal to the largest number of edges from the root to the most distant leaf node.

A similar concept in a binary tree is the depth of the tree. The depth of a node in a binary tree is the total number of edges from the root node to the target node. Similarly, the depth of a binary tree is the total number of edges from the root node to the most distant leaf node.

\section{\textbf{\color{blue}Threaded binary tree}}
In a threaded binary tree all the null printers which 
wasted the space in linked representation is converted 
into useful links called threads thus representation of a 
binary tree using these threads is called threaded 
binary tree.
\begin{center}
\includegraphics[width=.60\textwidth]{Threaded tree.jpg}

\end{center}

\section{\textbf{\color{blue}Equation}}

1.Infix Expression is A + B * C$\\
2.Prefix Expression is + A * B C$\\
3.Postfix Expression is A B C * +$\\

\section{\color{blue}Algorithm complexity}

\begin{center}  
\begin{tabular}{|l|c|r|t|}
\hline  
Algorithm &Best Case&Expected&New Col\\ \hline

Selection sort&O(N2)&O(N2)&A\\ \hline 
Merge sort &O(NlogN) &O(NlogN)&B\\ \hline 
Linear search& O(1)& O(N)&C\\ \hline 
Binary search &O(1)& O(logN)&D\\ \hline 
E&F&G&H\\ \hline 

\end{tabular}
\end{center}
Algorithmic complexity is a measure of how long an algorithm would take to complete given an input of size n. If an algorithm has to scale, it should compute the result within a finite and practical time bound even for large values of n. For this reason, complexity is calculated asymptotically as n approaches infinity.

\section{\textbf{\color{blue}Binary tree}}
 \begin{center}
     \includegraphics[width=\linewidth]{Tree2.jpg}
 \end{center}
 Binary search tree is a data structure that quickly allows us to maintain a sorted list of numbers.It is called a binary tree because each tree node has a maximum of two children.
 
\section{\color{blue}Analysis of sequential search}
To analyze searching algorithms, we need to decide 
on a basic unit of computation. Recall that this is 
typically the common step that must be repeated in 
order to solve the problem. For searching, it makes 
sense to count the number of comparisons performed. 
Each comparison may or may not discover the item 
we are looking for. In addition, we make another 
assumption here. The list of items is not ordered in any way. The items have been placed randomly into 
the list. In other words, the probability that the item 
we are looking for is in any particular position is 
exactly the same for each position of the list. 
If the item is not in the list, the only way to know it is 
to compare it against every item present. If there are 
\(n\) items, then the sequential search requires \(n\) 
comparisons to discover that the item is not there. In 
the case where the item is in the list, the analysis is 
not so straightforward. There are actually three 
different scenarios that can occur. In the best case we 
will find the item in the first place we look, at the 
beginning of the list. We will need only one 
comparison. In the worst case, we will not discover 
the item until the very last comparison, the nth 
comparison.
\section{\color{blue}Conclusion }
This paper covered the basics of data structures.
Data Structures is not just limited to Stack, 
Queues, and Linked Lists but is quite a vast area.
There are many more data structures which include  Graphs, Trees, etc. Each data 
structure has its own advantages and disadvantages 
and must be used according to the needs of the 
application.It is important to know how 
things work under the hood. Dynamic data structures 
require dynamic storage allocation and reclamation. 
This may be accomplished by the programmer or may 
be done implicitly by a high-level language. It is 
important to understand the fundamentals of storage 
management because these techniques have
significant impact on the behavior of programs. The
basic idea is to keep a pool of memory elements that 
may be used to store components of dynamic data 
structures when needed. Allocated storage may be 
returned to the pool when no longer needed. In this 
way, it may be used and reused. This contrasts sharply 
with static allocation, in which storage is dedicated
or the use of static data structures. It cannot then be 
reclaimed for other uses, even when no needed for the 
static data structure. As a result, dynamic allocation 
makes it possible to solve larger problems that might 
otherwise be storage-limited. Garbage collection and 
reference counters are two basic techniques for 
implementing storage management. 
\section{\color{blue}References}
1. Book of Data structures through C G. S Baluja.\\
2. Pieren Garry Department of computer science 
New York University.\\
3. Nick jones department of data mining Australia.\\
4. Wikipedia sequential search.\\

\end{document}